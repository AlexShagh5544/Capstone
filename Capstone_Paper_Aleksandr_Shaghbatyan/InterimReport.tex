\documentclass[a4paper,twocolumn,10.5pt]{article}
\usepackage[letterpaper,scale=0.875,right=1.3cm,top=2cm]{geometry} % 
\input{CIENCIADESARROLLO.sty}  

{\title{
\vspace*{1cm}
\Large{ Fraud Detection Tool \\
\large{\textcolor{carnelian}{\emph{Interim Report}}}
\\[0.2cm]}}}



\author[1,*]{\fontsize{10pt}{10pt}\selectfont \textbf{Aleksandr Shaghbatyan, Gurgen Hovakimyan}}

\begin{document}

\twocolumn
[
\begin{@twocolumnfalse}
\maketitle
\begin{flushleft}
    \textbf{Submission Date:} March 19, 2024\\
\end{flushleft}

\vspace*{0.3cm}





\renewcommand{\abstractname}{\textcolor{carnelian}{Progress Report}}
\begin{abstract}
\vspace*{0.5cm}
\fontsize{10pt}{10pt}\selectfont
\justify

Following the work plan of the project proposal, the work have  been started by defining the target audience. The identified target audience comprises financial risk managers and compliance services employees who encounter challenges due to the lack of coding knowledge. 

The next step was conducting a systematic literature review of projects and papers on similar topics. The utilization of machine learning algorithms in risk management and compliance has garnered significant interest in the last few years, resulting in sharp increase of scientific literature papers available on this subject. 

The subsequent step was finding appropriate dataset for the project. The dataset is called Synthetic Financial Dataset For Fraud Detection taken from Kaggle. The dataset is generated by the PaySim mobile money simulator. 

After conducting a thorough literature review and gathering sufficient information about the techniques used to accomplish such projects, the first and the second tabs dedicated to data cleaning and data visualization respectively were created. 

After uploading the dataset in the data cleaning tab it displays the number of missing values found in each column and offers a variety of methods to deal with those missing values like mean, median, mode, backward, and forward fill methods for each column separately. Nevertheless, the tab allows users to remove outliers of the numerical columns, and do data normalization/standardization. 

In the data visualization tab, users are prompted to select the type of visualization they wish to create such as a histogram, bar chart, line chart, scatter plot, or heatmap after uploading the dataset. Following the selection of the chart type, users can specify corresponding columns if needed and personalize the visualizations by making modifications such as color choice or adding hue to scatter plots.

Following the creation of the tabs, the subsequent step was creating a template in the Overleaf web page for drafting the paper using Latex. After getting acquainted with all the necessary tools of Latex, the abstract and introduction sections were composed. Moreover, drawing from the previously mentioned reading materials the literature review section has also been written. 

\end{abstract}



\vspace*{0.5cm}
\headrule
\vspace*{0.5cm}
\renewcommand{\abstractname}{\textcolor{carnelian}{Updated work plan}}
\begin{abstract}
\vspace*{0.5cm}
\fontsize{10pt}{10pt}\selectfont
\justify

Based on the current progress, the updated work plan will be as follows:

Phase 1: Creating model development tab
\begin{itemize}
    \item Design and train machine learning models for fraud detection
    \item Integrate model evolution metrics and visualization tools 
    \item Implement prediction functionality for new transactions 
    
\end{itemize}
 
Phase 2: Paper writing
\begin{itemize}
    \item Data cleaning section: Describe chosen technologies and libraries and include results for the tab in the project paper.
    \item Data visualization section: Write a description of the functionalities and results in the project paper 
    \item Model development section: Describe the model development process, evaluation metrics, prediction results, and used machine learning algorithms
\end{itemize}

Phase 3: Paper finalization

\begin{itemize}
    \item Include used dataset description 
    \item Based on final results include conclusion 
\end{itemize}

\end{abstract}


\end{@twocolumnfalse}]


\end{document} 
